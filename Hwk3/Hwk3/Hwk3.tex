
% Default to the notebook output style

    


% Inherit from the specified cell style.




    
\documentclass[11pt]{article}

    
    
    \usepackage[T1]{fontenc}
    % Nicer default font (+ math font) than Computer Modern for most use cases
    \usepackage{mathpazo}

    % Basic figure setup, for now with no caption control since it's done
    % automatically by Pandoc (which extracts ![](path) syntax from Markdown).
    \usepackage{graphicx}
    % We will generate all images so they have a width \maxwidth. This means
    % that they will get their normal width if they fit onto the page, but
    % are scaled down if they would overflow the margins.
    \makeatletter
    \def\maxwidth{\ifdim\Gin@nat@width>\linewidth\linewidth
    \else\Gin@nat@width\fi}
    \makeatother
    \let\Oldincludegraphics\includegraphics
    % Set max figure width to be 80% of text width, for now hardcoded.
    \renewcommand{\includegraphics}[1]{\Oldincludegraphics[width=.8\maxwidth]{#1}}
    % Ensure that by default, figures have no caption (until we provide a
    % proper Figure object with a Caption API and a way to capture that
    % in the conversion process - todo).
    \usepackage{caption}
    \DeclareCaptionLabelFormat{nolabel}{}
    \captionsetup{labelformat=nolabel}

    \usepackage{adjustbox} % Used to constrain images to a maximum size 
    \usepackage{xcolor} % Allow colors to be defined
    \usepackage{enumerate} % Needed for markdown enumerations to work
    \usepackage{geometry} % Used to adjust the document margins
    \usepackage{amsmath} % Equations
    \usepackage{amssymb} % Equations
    \usepackage{textcomp} % defines textquotesingle
    % Hack from http://tex.stackexchange.com/a/47451/13684:
    \AtBeginDocument{%
        \def\PYZsq{\textquotesingle}% Upright quotes in Pygmentized code
    }
    \usepackage{upquote} % Upright quotes for verbatim code
    \usepackage{eurosym} % defines \euro
    \usepackage[mathletters]{ucs} % Extended unicode (utf-8) support
    \usepackage[utf8x]{inputenc} % Allow utf-8 characters in the tex document
    \usepackage{fancyvrb} % verbatim replacement that allows latex
    \usepackage{grffile} % extends the file name processing of package graphics 
                         % to support a larger range 
    % The hyperref package gives us a pdf with properly built
    % internal navigation ('pdf bookmarks' for the table of contents,
    % internal cross-reference links, web links for URLs, etc.)
    \usepackage{hyperref}
    \usepackage{longtable} % longtable support required by pandoc >1.10
    \usepackage{booktabs}  % table support for pandoc > 1.12.2
    \usepackage[inline]{enumitem} % IRkernel/repr support (it uses the enumerate* environment)
    \usepackage[normalem]{ulem} % ulem is needed to support strikethroughs (\sout)
                                % normalem makes italics be italics, not underlines
    

    
    
    % Colors for the hyperref package
    \definecolor{urlcolor}{rgb}{0,.145,.698}
    \definecolor{linkcolor}{rgb}{.71,0.21,0.01}
    \definecolor{citecolor}{rgb}{.12,.54,.11}

    % ANSI colors
    \definecolor{ansi-black}{HTML}{3E424D}
    \definecolor{ansi-black-intense}{HTML}{282C36}
    \definecolor{ansi-red}{HTML}{E75C58}
    \definecolor{ansi-red-intense}{HTML}{B22B31}
    \definecolor{ansi-green}{HTML}{00A250}
    \definecolor{ansi-green-intense}{HTML}{007427}
    \definecolor{ansi-yellow}{HTML}{DDB62B}
    \definecolor{ansi-yellow-intense}{HTML}{B27D12}
    \definecolor{ansi-blue}{HTML}{208FFB}
    \definecolor{ansi-blue-intense}{HTML}{0065CA}
    \definecolor{ansi-magenta}{HTML}{D160C4}
    \definecolor{ansi-magenta-intense}{HTML}{A03196}
    \definecolor{ansi-cyan}{HTML}{60C6C8}
    \definecolor{ansi-cyan-intense}{HTML}{258F8F}
    \definecolor{ansi-white}{HTML}{C5C1B4}
    \definecolor{ansi-white-intense}{HTML}{A1A6B2}

    % commands and environments needed by pandoc snippets
    % extracted from the output of `pandoc -s`
    \providecommand{\tightlist}{%
      \setlength{\itemsep}{0pt}\setlength{\parskip}{0pt}}
    \DefineVerbatimEnvironment{Highlighting}{Verbatim}{commandchars=\\\{\}}
    % Add ',fontsize=\small' for more characters per line
    \newenvironment{Shaded}{}{}
    \newcommand{\KeywordTok}[1]{\textcolor[rgb]{0.00,0.44,0.13}{\textbf{{#1}}}}
    \newcommand{\DataTypeTok}[1]{\textcolor[rgb]{0.56,0.13,0.00}{{#1}}}
    \newcommand{\DecValTok}[1]{\textcolor[rgb]{0.25,0.63,0.44}{{#1}}}
    \newcommand{\BaseNTok}[1]{\textcolor[rgb]{0.25,0.63,0.44}{{#1}}}
    \newcommand{\FloatTok}[1]{\textcolor[rgb]{0.25,0.63,0.44}{{#1}}}
    \newcommand{\CharTok}[1]{\textcolor[rgb]{0.25,0.44,0.63}{{#1}}}
    \newcommand{\StringTok}[1]{\textcolor[rgb]{0.25,0.44,0.63}{{#1}}}
    \newcommand{\CommentTok}[1]{\textcolor[rgb]{0.38,0.63,0.69}{\textit{{#1}}}}
    \newcommand{\OtherTok}[1]{\textcolor[rgb]{0.00,0.44,0.13}{{#1}}}
    \newcommand{\AlertTok}[1]{\textcolor[rgb]{1.00,0.00,0.00}{\textbf{{#1}}}}
    \newcommand{\FunctionTok}[1]{\textcolor[rgb]{0.02,0.16,0.49}{{#1}}}
    \newcommand{\RegionMarkerTok}[1]{{#1}}
    \newcommand{\ErrorTok}[1]{\textcolor[rgb]{1.00,0.00,0.00}{\textbf{{#1}}}}
    \newcommand{\NormalTok}[1]{{#1}}
    
    % Additional commands for more recent versions of Pandoc
    \newcommand{\ConstantTok}[1]{\textcolor[rgb]{0.53,0.00,0.00}{{#1}}}
    \newcommand{\SpecialCharTok}[1]{\textcolor[rgb]{0.25,0.44,0.63}{{#1}}}
    \newcommand{\VerbatimStringTok}[1]{\textcolor[rgb]{0.25,0.44,0.63}{{#1}}}
    \newcommand{\SpecialStringTok}[1]{\textcolor[rgb]{0.73,0.40,0.53}{{#1}}}
    \newcommand{\ImportTok}[1]{{#1}}
    \newcommand{\DocumentationTok}[1]{\textcolor[rgb]{0.73,0.13,0.13}{\textit{{#1}}}}
    \newcommand{\AnnotationTok}[1]{\textcolor[rgb]{0.38,0.63,0.69}{\textbf{\textit{{#1}}}}}
    \newcommand{\CommentVarTok}[1]{\textcolor[rgb]{0.38,0.63,0.69}{\textbf{\textit{{#1}}}}}
    \newcommand{\VariableTok}[1]{\textcolor[rgb]{0.10,0.09,0.49}{{#1}}}
    \newcommand{\ControlFlowTok}[1]{\textcolor[rgb]{0.00,0.44,0.13}{\textbf{{#1}}}}
    \newcommand{\OperatorTok}[1]{\textcolor[rgb]{0.40,0.40,0.40}{{#1}}}
    \newcommand{\BuiltInTok}[1]{{#1}}
    \newcommand{\ExtensionTok}[1]{{#1}}
    \newcommand{\PreprocessorTok}[1]{\textcolor[rgb]{0.74,0.48,0.00}{{#1}}}
    \newcommand{\AttributeTok}[1]{\textcolor[rgb]{0.49,0.56,0.16}{{#1}}}
    \newcommand{\InformationTok}[1]{\textcolor[rgb]{0.38,0.63,0.69}{\textbf{\textit{{#1}}}}}
    \newcommand{\WarningTok}[1]{\textcolor[rgb]{0.38,0.63,0.69}{\textbf{\textit{{#1}}}}}
    
    
    % Define a nice break command that doesn't care if a line doesn't already
    % exist.
    \def\br{\hspace*{\fill} \\* }
    % Math Jax compatability definitions
    \def\gt{>}
    \def\lt{<}
    % Document parameters
    \title{Hwk3}
    
    
    

    % Pygments definitions
    
\makeatletter
\def\PY@reset{\let\PY@it=\relax \let\PY@bf=\relax%
    \let\PY@ul=\relax \let\PY@tc=\relax%
    \let\PY@bc=\relax \let\PY@ff=\relax}
\def\PY@tok#1{\csname PY@tok@#1\endcsname}
\def\PY@toks#1+{\ifx\relax#1\empty\else%
    \PY@tok{#1}\expandafter\PY@toks\fi}
\def\PY@do#1{\PY@bc{\PY@tc{\PY@ul{%
    \PY@it{\PY@bf{\PY@ff{#1}}}}}}}
\def\PY#1#2{\PY@reset\PY@toks#1+\relax+\PY@do{#2}}

\expandafter\def\csname PY@tok@w\endcsname{\def\PY@tc##1{\textcolor[rgb]{0.73,0.73,0.73}{##1}}}
\expandafter\def\csname PY@tok@c\endcsname{\let\PY@it=\textit\def\PY@tc##1{\textcolor[rgb]{0.25,0.50,0.50}{##1}}}
\expandafter\def\csname PY@tok@cp\endcsname{\def\PY@tc##1{\textcolor[rgb]{0.74,0.48,0.00}{##1}}}
\expandafter\def\csname PY@tok@k\endcsname{\let\PY@bf=\textbf\def\PY@tc##1{\textcolor[rgb]{0.00,0.50,0.00}{##1}}}
\expandafter\def\csname PY@tok@kp\endcsname{\def\PY@tc##1{\textcolor[rgb]{0.00,0.50,0.00}{##1}}}
\expandafter\def\csname PY@tok@kt\endcsname{\def\PY@tc##1{\textcolor[rgb]{0.69,0.00,0.25}{##1}}}
\expandafter\def\csname PY@tok@o\endcsname{\def\PY@tc##1{\textcolor[rgb]{0.40,0.40,0.40}{##1}}}
\expandafter\def\csname PY@tok@ow\endcsname{\let\PY@bf=\textbf\def\PY@tc##1{\textcolor[rgb]{0.67,0.13,1.00}{##1}}}
\expandafter\def\csname PY@tok@nb\endcsname{\def\PY@tc##1{\textcolor[rgb]{0.00,0.50,0.00}{##1}}}
\expandafter\def\csname PY@tok@nf\endcsname{\def\PY@tc##1{\textcolor[rgb]{0.00,0.00,1.00}{##1}}}
\expandafter\def\csname PY@tok@nc\endcsname{\let\PY@bf=\textbf\def\PY@tc##1{\textcolor[rgb]{0.00,0.00,1.00}{##1}}}
\expandafter\def\csname PY@tok@nn\endcsname{\let\PY@bf=\textbf\def\PY@tc##1{\textcolor[rgb]{0.00,0.00,1.00}{##1}}}
\expandafter\def\csname PY@tok@ne\endcsname{\let\PY@bf=\textbf\def\PY@tc##1{\textcolor[rgb]{0.82,0.25,0.23}{##1}}}
\expandafter\def\csname PY@tok@nv\endcsname{\def\PY@tc##1{\textcolor[rgb]{0.10,0.09,0.49}{##1}}}
\expandafter\def\csname PY@tok@no\endcsname{\def\PY@tc##1{\textcolor[rgb]{0.53,0.00,0.00}{##1}}}
\expandafter\def\csname PY@tok@nl\endcsname{\def\PY@tc##1{\textcolor[rgb]{0.63,0.63,0.00}{##1}}}
\expandafter\def\csname PY@tok@ni\endcsname{\let\PY@bf=\textbf\def\PY@tc##1{\textcolor[rgb]{0.60,0.60,0.60}{##1}}}
\expandafter\def\csname PY@tok@na\endcsname{\def\PY@tc##1{\textcolor[rgb]{0.49,0.56,0.16}{##1}}}
\expandafter\def\csname PY@tok@nt\endcsname{\let\PY@bf=\textbf\def\PY@tc##1{\textcolor[rgb]{0.00,0.50,0.00}{##1}}}
\expandafter\def\csname PY@tok@nd\endcsname{\def\PY@tc##1{\textcolor[rgb]{0.67,0.13,1.00}{##1}}}
\expandafter\def\csname PY@tok@s\endcsname{\def\PY@tc##1{\textcolor[rgb]{0.73,0.13,0.13}{##1}}}
\expandafter\def\csname PY@tok@sd\endcsname{\let\PY@it=\textit\def\PY@tc##1{\textcolor[rgb]{0.73,0.13,0.13}{##1}}}
\expandafter\def\csname PY@tok@si\endcsname{\let\PY@bf=\textbf\def\PY@tc##1{\textcolor[rgb]{0.73,0.40,0.53}{##1}}}
\expandafter\def\csname PY@tok@se\endcsname{\let\PY@bf=\textbf\def\PY@tc##1{\textcolor[rgb]{0.73,0.40,0.13}{##1}}}
\expandafter\def\csname PY@tok@sr\endcsname{\def\PY@tc##1{\textcolor[rgb]{0.73,0.40,0.53}{##1}}}
\expandafter\def\csname PY@tok@ss\endcsname{\def\PY@tc##1{\textcolor[rgb]{0.10,0.09,0.49}{##1}}}
\expandafter\def\csname PY@tok@sx\endcsname{\def\PY@tc##1{\textcolor[rgb]{0.00,0.50,0.00}{##1}}}
\expandafter\def\csname PY@tok@m\endcsname{\def\PY@tc##1{\textcolor[rgb]{0.40,0.40,0.40}{##1}}}
\expandafter\def\csname PY@tok@gh\endcsname{\let\PY@bf=\textbf\def\PY@tc##1{\textcolor[rgb]{0.00,0.00,0.50}{##1}}}
\expandafter\def\csname PY@tok@gu\endcsname{\let\PY@bf=\textbf\def\PY@tc##1{\textcolor[rgb]{0.50,0.00,0.50}{##1}}}
\expandafter\def\csname PY@tok@gd\endcsname{\def\PY@tc##1{\textcolor[rgb]{0.63,0.00,0.00}{##1}}}
\expandafter\def\csname PY@tok@gi\endcsname{\def\PY@tc##1{\textcolor[rgb]{0.00,0.63,0.00}{##1}}}
\expandafter\def\csname PY@tok@gr\endcsname{\def\PY@tc##1{\textcolor[rgb]{1.00,0.00,0.00}{##1}}}
\expandafter\def\csname PY@tok@ge\endcsname{\let\PY@it=\textit}
\expandafter\def\csname PY@tok@gs\endcsname{\let\PY@bf=\textbf}
\expandafter\def\csname PY@tok@gp\endcsname{\let\PY@bf=\textbf\def\PY@tc##1{\textcolor[rgb]{0.00,0.00,0.50}{##1}}}
\expandafter\def\csname PY@tok@go\endcsname{\def\PY@tc##1{\textcolor[rgb]{0.53,0.53,0.53}{##1}}}
\expandafter\def\csname PY@tok@gt\endcsname{\def\PY@tc##1{\textcolor[rgb]{0.00,0.27,0.87}{##1}}}
\expandafter\def\csname PY@tok@err\endcsname{\def\PY@bc##1{\setlength{\fboxsep}{0pt}\fcolorbox[rgb]{1.00,0.00,0.00}{1,1,1}{\strut ##1}}}
\expandafter\def\csname PY@tok@kc\endcsname{\let\PY@bf=\textbf\def\PY@tc##1{\textcolor[rgb]{0.00,0.50,0.00}{##1}}}
\expandafter\def\csname PY@tok@kd\endcsname{\let\PY@bf=\textbf\def\PY@tc##1{\textcolor[rgb]{0.00,0.50,0.00}{##1}}}
\expandafter\def\csname PY@tok@kn\endcsname{\let\PY@bf=\textbf\def\PY@tc##1{\textcolor[rgb]{0.00,0.50,0.00}{##1}}}
\expandafter\def\csname PY@tok@kr\endcsname{\let\PY@bf=\textbf\def\PY@tc##1{\textcolor[rgb]{0.00,0.50,0.00}{##1}}}
\expandafter\def\csname PY@tok@bp\endcsname{\def\PY@tc##1{\textcolor[rgb]{0.00,0.50,0.00}{##1}}}
\expandafter\def\csname PY@tok@fm\endcsname{\def\PY@tc##1{\textcolor[rgb]{0.00,0.00,1.00}{##1}}}
\expandafter\def\csname PY@tok@vc\endcsname{\def\PY@tc##1{\textcolor[rgb]{0.10,0.09,0.49}{##1}}}
\expandafter\def\csname PY@tok@vg\endcsname{\def\PY@tc##1{\textcolor[rgb]{0.10,0.09,0.49}{##1}}}
\expandafter\def\csname PY@tok@vi\endcsname{\def\PY@tc##1{\textcolor[rgb]{0.10,0.09,0.49}{##1}}}
\expandafter\def\csname PY@tok@vm\endcsname{\def\PY@tc##1{\textcolor[rgb]{0.10,0.09,0.49}{##1}}}
\expandafter\def\csname PY@tok@sa\endcsname{\def\PY@tc##1{\textcolor[rgb]{0.73,0.13,0.13}{##1}}}
\expandafter\def\csname PY@tok@sb\endcsname{\def\PY@tc##1{\textcolor[rgb]{0.73,0.13,0.13}{##1}}}
\expandafter\def\csname PY@tok@sc\endcsname{\def\PY@tc##1{\textcolor[rgb]{0.73,0.13,0.13}{##1}}}
\expandafter\def\csname PY@tok@dl\endcsname{\def\PY@tc##1{\textcolor[rgb]{0.73,0.13,0.13}{##1}}}
\expandafter\def\csname PY@tok@s2\endcsname{\def\PY@tc##1{\textcolor[rgb]{0.73,0.13,0.13}{##1}}}
\expandafter\def\csname PY@tok@sh\endcsname{\def\PY@tc##1{\textcolor[rgb]{0.73,0.13,0.13}{##1}}}
\expandafter\def\csname PY@tok@s1\endcsname{\def\PY@tc##1{\textcolor[rgb]{0.73,0.13,0.13}{##1}}}
\expandafter\def\csname PY@tok@mb\endcsname{\def\PY@tc##1{\textcolor[rgb]{0.40,0.40,0.40}{##1}}}
\expandafter\def\csname PY@tok@mf\endcsname{\def\PY@tc##1{\textcolor[rgb]{0.40,0.40,0.40}{##1}}}
\expandafter\def\csname PY@tok@mh\endcsname{\def\PY@tc##1{\textcolor[rgb]{0.40,0.40,0.40}{##1}}}
\expandafter\def\csname PY@tok@mi\endcsname{\def\PY@tc##1{\textcolor[rgb]{0.40,0.40,0.40}{##1}}}
\expandafter\def\csname PY@tok@il\endcsname{\def\PY@tc##1{\textcolor[rgb]{0.40,0.40,0.40}{##1}}}
\expandafter\def\csname PY@tok@mo\endcsname{\def\PY@tc##1{\textcolor[rgb]{0.40,0.40,0.40}{##1}}}
\expandafter\def\csname PY@tok@ch\endcsname{\let\PY@it=\textit\def\PY@tc##1{\textcolor[rgb]{0.25,0.50,0.50}{##1}}}
\expandafter\def\csname PY@tok@cm\endcsname{\let\PY@it=\textit\def\PY@tc##1{\textcolor[rgb]{0.25,0.50,0.50}{##1}}}
\expandafter\def\csname PY@tok@cpf\endcsname{\let\PY@it=\textit\def\PY@tc##1{\textcolor[rgb]{0.25,0.50,0.50}{##1}}}
\expandafter\def\csname PY@tok@c1\endcsname{\let\PY@it=\textit\def\PY@tc##1{\textcolor[rgb]{0.25,0.50,0.50}{##1}}}
\expandafter\def\csname PY@tok@cs\endcsname{\let\PY@it=\textit\def\PY@tc##1{\textcolor[rgb]{0.25,0.50,0.50}{##1}}}

\def\PYZbs{\char`\\}
\def\PYZus{\char`\_}
\def\PYZob{\char`\{}
\def\PYZcb{\char`\}}
\def\PYZca{\char`\^}
\def\PYZam{\char`\&}
\def\PYZlt{\char`\<}
\def\PYZgt{\char`\>}
\def\PYZsh{\char`\#}
\def\PYZpc{\char`\%}
\def\PYZdl{\char`\$}
\def\PYZhy{\char`\-}
\def\PYZsq{\char`\'}
\def\PYZdq{\char`\"}
\def\PYZti{\char`\~}
% for compatibility with earlier versions
\def\PYZat{@}
\def\PYZlb{[}
\def\PYZrb{]}
\makeatother


    % Exact colors from NB
    \definecolor{incolor}{rgb}{0.0, 0.0, 0.5}
    \definecolor{outcolor}{rgb}{0.545, 0.0, 0.0}



    
    % Prevent overflowing lines due to hard-to-break entities
    \sloppy 
    % Setup hyperref package
    \hypersetup{
      breaklinks=true,  % so long urls are correctly broken across lines
      colorlinks=true,
      urlcolor=urlcolor,
      linkcolor=linkcolor,
      citecolor=citecolor,
      }
    % Slightly bigger margins than the latex defaults
    
    \geometry{verbose,tmargin=1in,bmargin=1in,lmargin=1in,rmargin=1in}
    
    

    \begin{document}
    
    
    \maketitle
    
    

    
    \subsubsection{Q.1}\label{q.1}

\textbf{\texorpdfstring{Show that if X has linearly independent
columns, then \(X^TX\) is invertible, and \(X^+ = (X^TX)^{-1}X^T\).}}\\
Do a SVD on X:\\
\(X=QDP^T\), where
\(Q\subset \mathbb{R} ^{m\times m}, P\subset \mathbb{R} ^{q\times q}, D\subset \mathbb{R} ^{m\times q}\).
Both \(Q\) and \(P\) are orthonormal matrices. \(D\) is full column as
well and \(\lambda_i >0, \forall i\).
\(D = \begin{bmatrix}\lambda_1 & 0 & \cdots & 0 \\ 0 & \lambda_2 & \cdots & \vdots \\ \vdots & & \ddots & \vdots \\ 0 & \cdots & 0 & \lambda_q\\ \vdots & \ddots & & 0 \\ \vdots & & \ddots &\vdots \\ 0 & \cdots & \cdots & 0 \end{bmatrix}\)\\
So, \(X^TX=PD^TQ^TQDP^T = PD^TDP^T\), which implies
\(|\det(X^TX)|=|det(P)||det(D^TD)||det(P^T)|\). Meanwhile, the
determinant of orthonormal matrices are either 1 or -1, and
\(det(D^TD) = \Pi_{i=1}^q \lambda_i^2 \neq 0\). Thus
\(\det(X^TX)\neq 0\). We also know that \(det(A) \neq 0 \iff A\) is
invertible. Thus, \(X^TX\) is invertible.
\\
(A). Because \(XX^+\) is invertible, we have \(XX^+ = (X^+)^TX^T\)\\
(B).\(XX^+X = X \Rightarrow (X^+)^TX^TX = X \Rightarrow (X^+)^TX^TX(X^TX)^{-1} = X(X^TX)^{-1} \Rightarrow (X^+)^T = X(X^TX)^{-1} \Rightarrow X^+ = (X^TX)^{-1}X^T\)
\\\\
\textbf{\texorpdfstring{Show that if X has linearly independent
rows, then \(XX^T\) is invertible, and \(X^+ = X^T(X^TX)^{-1}\).}}\\
Do a SVD on X:\\
\(X=QDP^T\), where
\(Q\subset \mathbb{R} ^{m\times m}, P\subset \mathbb{R} ^{q\times q}, D\subset \mathbb{R} ^{m\times q}\).
Both \(Q\) and \(P\) are orthonormal matrices. \(D\) is full row as well
and \(\lambda_i >0, \forall i\).
\(D = \begin{bmatrix}\lambda_1 & 0 & \cdots & 0 & \cdots & 0 \\ 0 & \lambda_2 & \cdots & \vdots & & \vdots \\ \vdots & & \ddots & \vdots & & \vdots \\ 0 & \cdots & 0 & \lambda_m & \cdots & 0\end{bmatrix}\)\\
So, \(XX^T=QDP^TPD^TQ^T = QDD^TQ^T\), which implies
\(|\det(X^TX)|=|det(Q)||det(DD^T)||det(Q^T)|\). Meanwhile, the
determinant of orthonormal matrices are either 1 or -1, and
\(det(DD^T) = \Pi_{i=1}^m \lambda_i^2 \neq 0\). Thus
\(\det(X^TX)\neq 0\). We also know that \(det(A) \neq 0 \iff A\) is
invertible. Thus, \(XX^T\) is invertible.
\\
(C). Because \(X^+X\) is invertible, we have \(X^+X = X^T(X^+)^T\)\\
(D).
\(XX^+X = X \Rightarrow XX^T(X^+)^T = X \Rightarrow (X^TX)^{-1}XX^T(X^+)^T = (X^TX)^{-1}X \Rightarrow (X^+)^T = (X^TX)^{-1}X \Rightarrow X^+ = X^T(X^TX)^{-1}\)

    \begin{Verbatim}[commandchars=\\\{\}]
{\color{incolor}In [{\color{incolor}2}]:} \PY{k+kn}{import} \PY{n+nn}{numpy} \PY{k}{as} \PY{n+nn}{np}
        \PY{k+kn}{import} \PY{n+nn}{matplotlib}\PY{n+nn}{.}\PY{n+nn}{pyplot} \PY{k}{as} \PY{n+nn}{plt}
        \PY{k+kn}{import} \PY{n+nn}{matplotlib} \PY{k}{as} \PY{n+nn}{mpl}
\end{Verbatim}

    \subsubsection{Q.2 Let f(x, y) = − log(1 − x − y) − log x − log y with
domain D = \{(x, y) : x + y \textless{} 1, x \textgreater{} 0, y
\textgreater{}
0\}.}\label{q.2-let-fx-y-log1-x-y-log-x-log-y-with-domain-d-x-y-x-y-1-x-0-y-0.}

\textbf{(a) Find the gradient and the Hessian of f on
paper.}\\

    \(\nabla f(x,y) = (\frac{\partial f(x,y)}{\partial x},\frac{\partial f(x,y)}{\partial y})^T = (\frac{1}{1-x-y}-\frac{1}{x}, \frac{1}{1-x-y}-\frac{1}{y})^T\)

\(H = \begin{bmatrix} \frac{\partial^2 f(x,y)}{\partial x^2} & \frac{\partial^2 f(x,y)}{\partial x \partial y}\\ \frac{\partial^2 f(x,y)}{\partial y \partial x} & \frac{\partial^2 f(x,y)}{\partial y^2} \end{bmatrix} =\begin{bmatrix} \frac{1}{(1-x-y)^2}+\frac{1}{x^2} & \frac{1}{(1-x-y)^2} \\ \frac{1}{(1-x-y)^2} & \frac{1}{(1-x-y)^2}+\frac{1}{y^2} \end{bmatrix}\)

    \paragraph{(b) Begin with an initial point in w\_0\_ ∈ D with η = 1 and
estimate the global minimum of f using the Gradient descent method,
which will provide you with points w1, w2, . . . ,. Report your initial
point w0 and η of your choice. Draw a graph that shows the trajectory
followed by the points at each iteration. Also, plot the energies f(w0),
f(w1), . . . , achieved by the points at each iteration. Note: During
the iterations, your point may ``jump'' out of D where f is undefined.
If that happens, change your initial starting point and/or
η.}\label{b-begin-with-an-initial-point-in-w_0_-d-with-ux3b7-1-and-estimate-the-global-minimum-of-f-using-the-gradient-descent-method-which-will-provide-you-with-points-w1-w2-.-.-.-.-report-your-initial-point-w0-and-ux3b7-of-your-choice.-draw-a-graph-that-shows-the-trajectory-followed-by-the-points-at-each-iteration.-also-plot-the-energies-fw0-fw1-.-.-.-achieved-by-the-points-at-each-iteration.-note-during-the-iterations-your-point-may-jump-out-of-d-where-f-is-undefined.-if-that-happens-change-your-initial-starting-point-andor-ux3b7.}

    \begin{Verbatim}[commandchars=\\\{\}]
{\color{incolor}In [{\color{incolor}3}]:} \PY{n}{np}\PY{o}{.}\PY{n}{random}\PY{o}{.}\PY{n}{seed}\PY{p}{(}\PY{l+m+mi}{45}\PY{p}{)}
        \PY{c+c1}{\PYZsh{} define a loss function}
        \PY{k}{def} \PY{n+nf}{losfunction} \PY{p}{(}\PY{n}{w}\PY{p}{)}\PY{p}{:}
            \PY{n}{loss} \PY{o}{=} \PY{o}{\PYZhy{}}\PY{n}{np}\PY{o}{.}\PY{n}{log}\PY{p}{(}\PY{l+m+mi}{1}\PY{o}{\PYZhy{}}\PY{n}{np}\PY{o}{.}\PY{n}{sum}\PY{p}{(}\PY{n}{w}\PY{p}{)}\PY{p}{)}\PY{o}{\PYZhy{}}\PY{n}{np}\PY{o}{.}\PY{n}{log}\PY{p}{(}\PY{n}{w}\PY{p}{[}\PY{l+m+mi}{0}\PY{p}{]}\PY{p}{)}\PY{o}{\PYZhy{}}\PY{n}{np}\PY{o}{.}\PY{n}{log}\PY{p}{(}\PY{n}{w}\PY{p}{[}\PY{l+m+mi}{1}\PY{p}{]}\PY{p}{)}
            \PY{k}{return} \PY{n}{loss}
        
        \PY{c+c1}{\PYZsh{}  setup}
        \PY{n}{w} \PY{o}{=} \PY{n+nb}{list}\PY{p}{(}\PY{p}{)}
        \PY{n}{loss} \PY{o}{=} \PY{n+nb}{list}\PY{p}{(}\PY{p}{)}
        \PY{n}{epsilon} \PY{o}{=} \PY{l+m+mf}{0.005}
        \PY{n}{eta} \PY{o}{=} \PY{l+m+mi}{1}
        \PY{n}{j}\PY{o}{=} \PY{l+m+mi}{0}
        \PY{c+c1}{\PYZsh{} give a dummy start point which will be deleted later so that index is working}
        \PY{n}{temp} \PY{o}{=} \PY{n}{np}\PY{o}{.}\PY{n}{array}\PY{p}{(}\PY{p}{[}\PY{l+m+mf}{0.1}\PY{p}{,}\PY{l+m+mf}{0.8}\PY{p}{]}\PY{p}{)}
        \PY{n}{temploss} \PY{o}{=} \PY{n}{losfunction}\PY{p}{(}\PY{n}{temp}\PY{p}{)}
        \PY{n}{w}\PY{o}{.}\PY{n}{append}\PY{p}{(}\PY{n}{temp}\PY{p}{)}
        \PY{n}{loss}\PY{o}{.}\PY{n}{append}\PY{p}{(}\PY{n}{temploss}\PY{p}{)}
        
        \PY{c+c1}{\PYZsh{} initialize the starting point w\PYZus{}0 in D}
        \PY{n}{temp} \PY{o}{=} \PY{n}{np}\PY{o}{.}\PY{n}{random}\PY{o}{.}\PY{n}{uniform}\PY{p}{(}\PY{l+m+mi}{0}\PY{p}{,}\PY{l+m+mi}{1}\PY{p}{,}\PY{l+m+mi}{2}\PY{p}{)}
        \PY{k}{while} \PY{p}{(}\PY{n}{np}\PY{o}{.}\PY{n}{sum}\PY{p}{(}\PY{n}{temp}\PY{p}{)}\PY{o}{\PYZgt{}}\PY{o}{=}\PY{l+m+mi}{1}\PY{p}{)}\PY{p}{:}
                \PY{n}{temp} \PY{o}{=} \PY{n}{np}\PY{o}{.}\PY{n}{random}\PY{o}{.}\PY{n}{uniform}\PY{p}{(}\PY{l+m+mi}{0}\PY{p}{,}\PY{l+m+mi}{1}\PY{p}{,}\PY{l+m+mi}{2}\PY{p}{)}
        \PY{c+c1}{\PYZsh{} append the first point in w}
        \PY{n}{w}\PY{o}{.}\PY{n}{append}\PY{p}{(}\PY{n}{temp}\PY{p}{)}
        \PY{n}{temploss} \PY{o}{=} \PY{n}{losfunction}\PY{p}{(}\PY{n}{temp}\PY{p}{)}
        \PY{n}{loss}\PY{o}{.}\PY{n}{append}\PY{p}{(}\PY{n}{temploss}\PY{p}{)}
        
        \PY{c+c1}{\PYZsh{} the while loops ends when converge}
        \PY{k}{while} \PY{p}{(}\PY{n}{np}\PY{o}{.}\PY{n}{abs}\PY{p}{(}\PY{n}{loss}\PY{p}{[}\PY{o}{\PYZhy{}}\PY{l+m+mi}{1}\PY{p}{]}\PY{o}{\PYZhy{}}\PY{n}{loss}\PY{p}{[}\PY{o}{\PYZhy{}}\PY{l+m+mi}{2}\PY{p}{]}\PY{p}{)}\PY{o}{\PYZgt{}}\PY{o}{=}\PY{n}{epsilon}\PY{p}{)}\PY{p}{:}
            \PY{c+c1}{\PYZsh{} find the gradient}
            \PY{n}{g} \PY{o}{=} \PY{n}{np}\PY{o}{.}\PY{n}{array}\PY{p}{(}\PY{p}{[}\PY{p}{(}\PY{l+m+mi}{1}\PY{o}{/}\PY{p}{(}\PY{l+m+mi}{1}\PY{o}{\PYZhy{}}\PY{n}{np}\PY{o}{.}\PY{n}{sum}\PY{p}{(}\PY{n}{w}\PY{p}{[}\PY{o}{\PYZhy{}}\PY{l+m+mi}{1}\PY{p}{]}\PY{p}{)}\PY{p}{)}\PY{o}{\PYZhy{}}\PY{l+m+mi}{1}\PY{o}{/}\PY{n}{w}\PY{p}{[}\PY{o}{\PYZhy{}}\PY{l+m+mi}{1}\PY{p}{]}\PY{p}{[}\PY{l+m+mi}{0}\PY{p}{]}\PY{p}{)}\PY{p}{,}\PY{p}{(}\PY{l+m+mi}{1}\PY{o}{/}\PY{p}{(}\PY{l+m+mi}{1}\PY{o}{\PYZhy{}}\PY{n}{np}\PY{o}{.}\PY{n}{sum}\PY{p}{(}\PY{n}{w}\PY{p}{[}\PY{o}{\PYZhy{}}\PY{l+m+mi}{1}\PY{p}{]}\PY{p}{)}\PY{p}{)}\PY{o}{\PYZhy{}}\PY{l+m+mi}{1}\PY{o}{/}\PY{n}{w}\PY{p}{[}\PY{o}{\PYZhy{}}\PY{l+m+mi}{1}\PY{p}{]}\PY{p}{[}\PY{l+m+mi}{1}\PY{p}{]}\PY{p}{)}\PY{p}{]}\PY{p}{)}
            \PY{n}{temp} \PY{o}{=} \PY{n}{w}\PY{p}{[}\PY{o}{\PYZhy{}}\PY{l+m+mi}{1}\PY{p}{]}\PY{o}{\PYZhy{}}\PY{n}{eta}\PY{o}{*}\PY{n}{g}
            \PY{c+c1}{\PYZsh{} if counter k is larger than 5, the overall eta will be reduced by 1/2}
            \PY{n}{etanew} \PY{o}{=} \PY{n}{eta}
            \PY{n}{k} \PY{o}{=} \PY{l+m+mi}{0}
            \PY{c+c1}{\PYZsh{} if the temp is outside the D, will redo the while loop and add the counter}
            \PY{k}{while} \PY{p}{(}\PY{n+nb}{sum}\PY{p}{(}\PY{n}{temp}\PY{p}{)}\PY{o}{\PYZgt{}}\PY{o}{=}\PY{l+m+mi}{1} \PY{o+ow}{or} \PY{n}{temp}\PY{p}{[}\PY{l+m+mi}{0}\PY{p}{]}\PY{o}{\PYZlt{}}\PY{o}{=}\PY{l+m+mi}{0} \PY{o+ow}{or} \PY{n}{temp}\PY{p}{[}\PY{l+m+mi}{1}\PY{p}{]}\PY{o}{\PYZlt{}}\PY{o}{=}\PY{l+m+mi}{0}\PY{p}{)}\PY{p}{:}
                \PY{n}{etanew} \PY{o}{=} \PY{n}{etanew}\PY{o}{/}\PY{l+m+mi}{2}
                \PY{n}{temp} \PY{o}{=} \PY{n}{w}\PY{p}{[}\PY{o}{\PYZhy{}}\PY{l+m+mi}{1}\PY{p}{]}\PY{o}{\PYZhy{}}\PY{n}{etanew}\PY{o}{*}\PY{n}{g}
                \PY{n}{k} \PY{o}{+}\PY{o}{=} \PY{l+m+mi}{1}
                \PY{n}{j} \PY{o}{+}\PY{o}{=} \PY{l+m+mi}{1}
            \PY{n}{w}\PY{o}{.}\PY{n}{append}\PY{p}{(}\PY{n}{temp}\PY{p}{)}
            \PY{n}{temploss} \PY{o}{=} \PY{n}{losfunction}\PY{p}{(}\PY{n}{temp}\PY{p}{)}
            \PY{n}{loss}\PY{o}{.}\PY{n}{append}\PY{p}{(}\PY{n}{temploss}\PY{p}{)}
            
            \PY{k}{if} \PY{n}{k}\PY{o}{\PYZgt{}}\PY{o}{=}\PY{l+m+mi}{5}\PY{p}{:}
                \PY{n}{eta} \PY{o}{=} \PY{n}{eta}\PY{o}{/}\PY{l+m+mi}{2}
        \PY{c+c1}{\PYZsh{} delete the dummy point}
        \PY{k}{del}\PY{p}{(}\PY{n}{w}\PY{p}{[}\PY{l+m+mi}{0}\PY{p}{]}\PY{p}{)}
        \PY{k}{del}\PY{p}{(}\PY{n}{loss}\PY{p}{[}\PY{l+m+mi}{0}\PY{p}{]}\PY{p}{)}
        \PY{c+c1}{\PYZsh{} convert w to ndarray for plot}
        \PY{n}{w} \PY{o}{=} \PY{n}{np}\PY{o}{.}\PY{n}{asarray}\PY{p}{(}\PY{n}{w}\PY{p}{)}
        \PY{n}{w}\PY{p}{[}\PY{l+m+mi}{0}\PY{p}{]}
\end{Verbatim}

            \begin{Verbatim}[commandchars=\\\{\}]
{\color{outcolor}Out[{\color{outcolor}3}]:} array([ 0.2814473 ,  0.07728957])
\end{Verbatim}
        
    \paragraph{\texorpdfstring{{Q. Report your initial point w0 and η of
your
choice.}}{Q. Report your initial point w0 and η of your choice.}}\label{q.-report-your-initial-point-w0-and-ux3b7-of-your-choice.}

\paragraph{\texorpdfstring{{The intial point is at {[} 0.2814473 ,
0.07728957{]}. The intial eta is 1, but if the update is out of
boundaries, eta will temporarily shrink by 1/2 until the update is
within boundaries. Also, if it takes for more than 5 temporaty attempts
to shrink the eta for one iteration, the overall eta will be shrinked by
1/2.
}}{The intial point is at {[} 0.2814473 , 0.07728957{]}. The intial eta is 1, but if the update is out of boundaries, eta will temporarily shrink by 1/2 until the update is within boundaries. Also, if it takes for more than 5 temporaty attempts to shrink the eta for one iteration, the overall eta will be shrinked by 1/2. }}\label{the-intial-point-is-at-0.2814473-0.07728957.-the-intial-eta-is-1-but-if-the-update-is-out-of-boundaries-eta-will-temporarily-shrink-by-12-until-the-update-is-within-boundaries.-also-if-it-takes-for-more-than-5-temporaty-attempts-to-shrink-the-eta-for-one-iteration-the-overall-eta-will-be-shrinked-by-12.}

    \paragraph{\texorpdfstring{{Q. Draw a graph that shows the trajectory
followed by the points at each iteration.
}}{Q. Draw a graph that shows the trajectory followed by the points at each iteration. }}\label{q.-draw-a-graph-that-shows-the-trajectory-followed-by-the-points-at-each-iteration.}

    \begin{Verbatim}[commandchars=\\\{\}]
{\color{incolor}In [{\color{incolor}4}]:} \PY{n}{fig}\PY{p}{,} \PY{n}{ax} \PY{o}{=} \PY{n}{plt}\PY{o}{.}\PY{n}{subplots}\PY{p}{(}\PY{p}{)}
        \PY{n}{A} \PY{o}{=} \PY{n}{w}\PY{p}{[}\PY{p}{:}\PY{p}{,}\PY{l+m+mi}{0}\PY{p}{]}
        \PY{n}{B} \PY{o}{=} \PY{n}{w}\PY{p}{[}\PY{p}{:}\PY{p}{,}\PY{l+m+mi}{1}\PY{p}{]}
        \PY{n}{n} \PY{o}{=} \PY{n+nb}{range}\PY{p}{(}\PY{n+nb}{len}\PY{p}{(}\PY{n}{w}\PY{p}{)}\PY{p}{)}
        \PY{n}{plt}\PY{o}{.}\PY{n}{plot}\PY{p}{(}\PY{n}{A}\PY{p}{,}\PY{n}{B}\PY{p}{,}\PY{l+s+s1}{\PYZsq{}}\PY{l+s+s1}{y\PYZhy{}\PYZgt{}}\PY{l+s+s1}{\PYZsq{}}\PY{p}{)}
        \PY{n}{C} \PY{o}{=} \PY{n+nb}{list}\PY{p}{(}\PY{n}{A}\PY{p}{)}
        \PY{k}{for} \PY{n}{i}\PY{p}{,} \PY{n}{txt} \PY{o+ow}{in} \PY{n+nb}{enumerate}\PY{p}{(}\PY{n}{C}\PY{p}{)}\PY{p}{:}
            \PY{n}{ax}\PY{o}{.}\PY{n}{annotate}\PY{p}{(}\PY{n}{n}\PY{p}{[}\PY{n}{i}\PY{p}{]}\PY{p}{,} \PY{p}{(}\PY{n}{A}\PY{p}{[}\PY{n}{i}\PY{p}{]}\PY{p}{,}\PY{n}{B}\PY{p}{[}\PY{n}{i}\PY{p}{]}\PY{p}{)}\PY{p}{)}
        \PY{n}{plt}\PY{o}{.}\PY{n}{plot}\PY{p}{(}\PY{p}{[}\PY{l+m+mi}{0}\PY{p}{,}\PY{l+m+mi}{1}\PY{p}{]}\PY{p}{,}\PY{p}{[}\PY{l+m+mi}{1}\PY{p}{,}\PY{l+m+mi}{0}\PY{p}{]}\PY{p}{,}\PY{l+s+s1}{\PYZsq{}}\PY{l+s+s1}{k\PYZhy{}}\PY{l+s+s1}{\PYZsq{}}\PY{p}{)}
        \PY{n}{plt}\PY{o}{.}\PY{n}{plot}\PY{p}{(}\PY{p}{[}\PY{l+m+mi}{0}\PY{p}{,}\PY{l+m+mi}{1}\PY{p}{]}\PY{p}{,}\PY{p}{[}\PY{l+m+mi}{0}\PY{p}{,}\PY{l+m+mi}{0}\PY{p}{]}\PY{p}{,}\PY{l+s+s1}{\PYZsq{}}\PY{l+s+s1}{k\PYZhy{}}\PY{l+s+s1}{\PYZsq{}}\PY{p}{)}
        \PY{n}{plt}\PY{o}{.}\PY{n}{plot}\PY{p}{(}\PY{p}{[}\PY{l+m+mi}{0}\PY{p}{,}\PY{l+m+mi}{0}\PY{p}{]}\PY{p}{,}\PY{p}{[}\PY{l+m+mi}{0}\PY{p}{,}\PY{l+m+mi}{1}\PY{p}{]}\PY{p}{,}\PY{l+s+s1}{\PYZsq{}}\PY{l+s+s1}{k\PYZhy{}}\PY{l+s+s1}{\PYZsq{}}\PY{p}{)}
        \PY{n}{plt}\PY{o}{.}\PY{n}{xlabel}\PY{p}{(}\PY{l+s+s1}{\PYZsq{}}\PY{l+s+s1}{x}\PY{l+s+s1}{\PYZsq{}}\PY{p}{)}
        \PY{n}{plt}\PY{o}{.}\PY{n}{title}\PY{p}{(}\PY{l+s+s1}{\PYZsq{}}\PY{l+s+s1}{Trajectory followed by the points at each iteration (b)}\PY{l+s+s1}{\PYZsq{}}\PY{p}{)}
        \PY{n}{plt}\PY{o}{.}\PY{n}{ylabel}\PY{p}{(}\PY{l+s+s1}{\PYZsq{}}\PY{l+s+s1}{y}\PY{l+s+s1}{\PYZsq{}}\PY{p}{)}
        \PY{n}{ax}\PY{o}{.}\PY{n}{text}\PY{p}{(}\PY{o}{.}\PY{l+m+mi}{6}\PY{p}{,} \PY{o}{.}\PY{l+m+mi}{8}\PY{p}{,} \PY{l+s+s1}{\PYZsq{}}\PY{l+s+s1}{\PYZdl{}}\PY{l+s+s1}{\PYZbs{}}\PY{l+s+s1}{epsilon =5e\PYZhy{}3\PYZdl{}}\PY{l+s+s1}{\PYZsq{}}\PY{p}{,} \PY{n}{style}\PY{o}{=}\PY{l+s+s1}{\PYZsq{}}\PY{l+s+s1}{italic}\PY{l+s+s1}{\PYZsq{}}\PY{p}{,}
                \PY{n}{bbox}\PY{o}{=}\PY{p}{\PYZob{}}\PY{l+s+s1}{\PYZsq{}}\PY{l+s+s1}{facecolor}\PY{l+s+s1}{\PYZsq{}}\PY{p}{:}\PY{l+s+s1}{\PYZsq{}}\PY{l+s+s1}{red}\PY{l+s+s1}{\PYZsq{}}\PY{p}{,} \PY{l+s+s1}{\PYZsq{}}\PY{l+s+s1}{alpha}\PY{l+s+s1}{\PYZsq{}}\PY{p}{:}\PY{l+m+mf}{0.3}\PY{p}{,} \PY{l+s+s1}{\PYZsq{}}\PY{l+s+s1}{pad}\PY{l+s+s1}{\PYZsq{}}\PY{p}{:}\PY{l+m+mi}{10}\PY{p}{\PYZcb{}}\PY{p}{,}\PY{n}{fontsize} \PY{o}{=} \PY{l+m+mi}{10}\PY{p}{)}
        \PY{n}{plt}\PY{o}{.}\PY{n}{show}\PY{p}{(}\PY{p}{)}
\end{Verbatim}

    \begin{center}
    \adjustimage{max size={0.9\linewidth}{0.9\paperheight}}{output_12_0.png}
    \end{center}
    { \hspace*{\fill} \\}
    
    \paragraph{\texorpdfstring{{Q. Plot the energies f(w0), f(w1), . . . ,
achieved by the points at each iteration.
}}{Q. Plot the energies f(w0), f(w1), . . . , achieved by the points at each iteration. }}\label{q.-plot-the-energies-fw0-fw1-.-.-.-achieved-by-the-points-at-each-iteration.}

    \begin{Verbatim}[commandchars=\\\{\}]
{\color{incolor}In [{\color{incolor}5}]:} \PY{n}{fig}\PY{p}{,} \PY{n}{ax} \PY{o}{=} \PY{n}{plt}\PY{o}{.}\PY{n}{subplots}\PY{p}{(}\PY{p}{)}
        \PY{n}{A} \PY{o}{=} \PY{n}{loss}
        \PY{n}{n} \PY{o}{=} \PY{n+nb}{range}\PY{p}{(}\PY{n+nb}{len}\PY{p}{(}\PY{n}{A}\PY{p}{)}\PY{p}{)}
        \PY{n}{plt}\PY{o}{.}\PY{n}{plot}\PY{p}{(}\PY{n}{n}\PY{p}{,}\PY{n}{A}\PY{p}{,}\PY{l+s+s1}{\PYZsq{}}\PY{l+s+s1}{r\PYZhy{}\PYZgt{}}\PY{l+s+s1}{\PYZsq{}}\PY{p}{)}
        \PY{k}{for} \PY{n}{i}\PY{p}{,} \PY{n}{txt} \PY{o+ow}{in} \PY{n+nb}{enumerate}\PY{p}{(}\PY{n}{A}\PY{p}{)}\PY{p}{:}
            \PY{n}{ax}\PY{o}{.}\PY{n}{annotate}\PY{p}{(}\PY{n+nb}{round}\PY{p}{(}\PY{n}{txt}\PY{p}{,}\PY{l+m+mi}{2}\PY{p}{)}\PY{p}{,} \PY{p}{(}\PY{n}{n}\PY{p}{[}\PY{n}{i}\PY{p}{]}\PY{p}{,}\PY{n}{A}\PY{p}{[}\PY{n}{i}\PY{p}{]}\PY{p}{)}\PY{p}{)}
        \PY{n}{plt}\PY{o}{.}\PY{n}{xlabel}\PY{p}{(}\PY{l+s+s1}{\PYZsq{}}\PY{l+s+s1}{Iteration}\PY{l+s+s1}{\PYZsq{}}\PY{p}{)}
        \PY{n}{plt}\PY{o}{.}\PY{n}{title}\PY{p}{(}\PY{l+s+s1}{\PYZsq{}}\PY{l+s+s1}{Cost at each iteration (b)}\PY{l+s+s1}{\PYZsq{}}\PY{p}{)}
        \PY{n}{plt}\PY{o}{.}\PY{n}{ylabel}\PY{p}{(}\PY{l+s+s1}{\PYZsq{}}\PY{l+s+s1}{Cost}\PY{l+s+s1}{\PYZsq{}}\PY{p}{)}
        \PY{n}{plt}\PY{o}{.}\PY{n}{show}\PY{p}{(}\PY{p}{)}
\end{Verbatim}

    \begin{center}
    \adjustimage{max size={0.9\linewidth}{0.9\paperheight}}{output_14_0.png}
    \end{center}
    { \hspace*{\fill} \\}
    
    \begin{Verbatim}[commandchars=\\\{\}]
{\color{incolor}In [{\color{incolor}6}]:} \PY{c+c1}{\PYZsh{} converging point}
        \PY{n}{w}\PY{p}{[}\PY{o}{\PYZhy{}}\PY{l+m+mi}{1}\PY{p}{]}
\end{Verbatim}

            \begin{Verbatim}[commandchars=\\\{\}]
{\color{outcolor}Out[{\color{outcolor}6}]:} array([ 0.31665327,  0.31669632])
\end{Verbatim}
        
    \begin{Verbatim}[commandchars=\\\{\}]
{\color{incolor}In [{\color{incolor}7}]:} \PY{c+c1}{\PYZsh{} converging cost}
        \PY{n}{loss}\PY{p}{[}\PY{o}{\PYZhy{}}\PY{l+m+mi}{1}\PY{p}{]}
\end{Verbatim}

            \begin{Verbatim}[commandchars=\\\{\}]
{\color{outcolor}Out[{\color{outcolor}7}]:} 3.3031062823844062
\end{Verbatim}
        
    \paragraph{(c) Repeat part (b) using Newton's
method.}\label{c-repeat-part-b-using-newtons-method.}

    \begin{Verbatim}[commandchars=\\\{\}]
{\color{incolor}In [{\color{incolor}8}]:} \PY{n}{np}\PY{o}{.}\PY{n}{random}\PY{o}{.}\PY{n}{seed}\PY{p}{(}\PY{l+m+mi}{45}\PY{p}{)}
        \PY{k}{def} \PY{n+nf}{Hmatrix} \PY{p}{(}\PY{n}{w}\PY{p}{)}\PY{p}{:}
            \PY{n}{x} \PY{o}{=} \PY{n}{w}\PY{p}{[}\PY{l+m+mi}{0}\PY{p}{]}
            \PY{n}{y} \PY{o}{=} \PY{n}{w}\PY{p}{[}\PY{l+m+mi}{1}\PY{p}{]}
            \PY{n}{mat} \PY{o}{=} \PY{p}{[}\PY{p}{[}\PY{p}{(}\PY{l+m+mi}{1}\PY{o}{/}\PY{p}{(}\PY{l+m+mi}{1}\PY{o}{\PYZhy{}}\PY{n}{x}\PY{o}{\PYZhy{}}\PY{n}{y}\PY{p}{)}\PY{o}{*}\PY{o}{*}\PY{l+m+mi}{2}\PY{o}{+}\PY{l+m+mi}{1}\PY{o}{/}\PY{n}{x}\PY{o}{*}\PY{o}{*}\PY{l+m+mi}{2}\PY{p}{)}\PY{p}{,}\PY{p}{(}\PY{l+m+mi}{1}\PY{o}{/}\PY{p}{(}\PY{l+m+mi}{1}\PY{o}{\PYZhy{}}\PY{n}{x}\PY{o}{\PYZhy{}}\PY{n}{y}\PY{p}{)}\PY{o}{*}\PY{o}{*}\PY{l+m+mi}{2}\PY{p}{)}\PY{p}{]}\PY{p}{,}\PY{p}{[}\PY{p}{(}\PY{l+m+mi}{1}\PY{o}{/}\PY{p}{(}\PY{l+m+mi}{1}\PY{o}{\PYZhy{}}\PY{n}{x}\PY{o}{\PYZhy{}}\PY{n}{y}\PY{p}{)}\PY{o}{*}\PY{o}{*}\PY{l+m+mi}{2}\PY{p}{)}\PY{p}{,}\PY{p}{(}\PY{l+m+mi}{1}\PY{o}{/}\PY{p}{(}\PY{l+m+mi}{1}\PY{o}{\PYZhy{}}\PY{n}{x}\PY{o}{\PYZhy{}}\PY{n}{y}\PY{p}{)}\PY{o}{*}\PY{o}{*}\PY{l+m+mi}{2}\PY{o}{+}\PY{l+m+mi}{1}\PY{o}{/}\PY{n}{y}\PY{o}{*}\PY{o}{*}\PY{l+m+mi}{2}\PY{p}{)}\PY{p}{]}\PY{p}{]}
            \PY{n}{out} \PY{o}{=} \PY{n}{np}\PY{o}{.}\PY{n}{asarray}\PY{p}{(}\PY{n}{mat}\PY{p}{)}
            \PY{k}{return} \PY{n}{out}
        \PY{c+c1}{\PYZsh{}  setup}
        \PY{n}{w} \PY{o}{=} \PY{n+nb}{list}\PY{p}{(}\PY{p}{)}
        \PY{n}{loss} \PY{o}{=} \PY{n+nb}{list}\PY{p}{(}\PY{p}{)}
        \PY{n}{epsilon} \PY{o}{=} \PY{l+m+mf}{0.005}
        \PY{n}{eta} \PY{o}{=} \PY{l+m+mi}{1}
        \PY{n}{j}\PY{o}{=} \PY{l+m+mi}{0}
        \PY{c+c1}{\PYZsh{} give a dummy start point which will be deleted later so that index is working}
        \PY{n}{temp} \PY{o}{=} \PY{n}{np}\PY{o}{.}\PY{n}{array}\PY{p}{(}\PY{p}{[}\PY{l+m+mf}{0.1}\PY{p}{,}\PY{l+m+mf}{0.8}\PY{p}{]}\PY{p}{)}
        \PY{n}{temploss} \PY{o}{=} \PY{n}{losfunction}\PY{p}{(}\PY{n}{temp}\PY{p}{)}
        \PY{n}{w}\PY{o}{.}\PY{n}{append}\PY{p}{(}\PY{n}{temp}\PY{p}{)}
        \PY{n}{loss}\PY{o}{.}\PY{n}{append}\PY{p}{(}\PY{n}{temploss}\PY{p}{)}
        
        \PY{c+c1}{\PYZsh{} initialize the starting point w\PYZus{}0 in D}
        \PY{n}{temp} \PY{o}{=} \PY{n}{np}\PY{o}{.}\PY{n}{random}\PY{o}{.}\PY{n}{uniform}\PY{p}{(}\PY{l+m+mi}{0}\PY{p}{,}\PY{l+m+mi}{1}\PY{p}{,}\PY{l+m+mi}{2}\PY{p}{)}
        \PY{k}{while} \PY{p}{(}\PY{n}{np}\PY{o}{.}\PY{n}{sum}\PY{p}{(}\PY{n}{temp}\PY{p}{)}\PY{o}{\PYZgt{}}\PY{o}{=}\PY{l+m+mi}{1}\PY{p}{)}\PY{p}{:}
                \PY{n}{temp} \PY{o}{=} \PY{n}{np}\PY{o}{.}\PY{n}{random}\PY{o}{.}\PY{n}{uniform}\PY{p}{(}\PY{l+m+mi}{0}\PY{p}{,}\PY{l+m+mi}{1}\PY{p}{,}\PY{l+m+mi}{2}\PY{p}{)}
        \PY{c+c1}{\PYZsh{} append the first point in w}
        \PY{n}{w}\PY{o}{.}\PY{n}{append}\PY{p}{(}\PY{n}{temp}\PY{p}{)}
        \PY{n}{temploss} \PY{o}{=} \PY{n}{losfunction}\PY{p}{(}\PY{n}{temp}\PY{p}{)}
        \PY{n}{loss}\PY{o}{.}\PY{n}{append}\PY{p}{(}\PY{n}{temploss}\PY{p}{)}
        
        \PY{c+c1}{\PYZsh{} the while loops ends when converge}
        \PY{k}{while} \PY{p}{(}\PY{n}{np}\PY{o}{.}\PY{n}{abs}\PY{p}{(}\PY{n}{loss}\PY{p}{[}\PY{o}{\PYZhy{}}\PY{l+m+mi}{1}\PY{p}{]}\PY{o}{\PYZhy{}}\PY{n}{loss}\PY{p}{[}\PY{o}{\PYZhy{}}\PY{l+m+mi}{2}\PY{p}{]}\PY{p}{)}\PY{o}{\PYZgt{}}\PY{o}{=}\PY{n}{epsilon}\PY{p}{)}\PY{p}{:}
            \PY{c+c1}{\PYZsh{} find the gradient}
            \PY{n}{g} \PY{o}{=} \PY{n}{np}\PY{o}{.}\PY{n}{array}\PY{p}{(}\PY{p}{[}\PY{p}{(}\PY{l+m+mi}{1}\PY{o}{/}\PY{p}{(}\PY{l+m+mi}{1}\PY{o}{\PYZhy{}}\PY{n}{np}\PY{o}{.}\PY{n}{sum}\PY{p}{(}\PY{n}{w}\PY{p}{[}\PY{o}{\PYZhy{}}\PY{l+m+mi}{1}\PY{p}{]}\PY{p}{)}\PY{p}{)}\PY{o}{\PYZhy{}}\PY{l+m+mi}{1}\PY{o}{/}\PY{n}{w}\PY{p}{[}\PY{o}{\PYZhy{}}\PY{l+m+mi}{1}\PY{p}{]}\PY{p}{[}\PY{l+m+mi}{0}\PY{p}{]}\PY{p}{)}\PY{p}{,}\PY{p}{(}\PY{l+m+mi}{1}\PY{o}{/}\PY{p}{(}\PY{l+m+mi}{1}\PY{o}{\PYZhy{}}\PY{n}{np}\PY{o}{.}\PY{n}{sum}\PY{p}{(}\PY{n}{w}\PY{p}{[}\PY{o}{\PYZhy{}}\PY{l+m+mi}{1}\PY{p}{]}\PY{p}{)}\PY{p}{)}\PY{o}{\PYZhy{}}\PY{l+m+mi}{1}\PY{o}{/}\PY{n}{w}\PY{p}{[}\PY{o}{\PYZhy{}}\PY{l+m+mi}{1}\PY{p}{]}\PY{p}{[}\PY{l+m+mi}{1}\PY{p}{]}\PY{p}{)}\PY{p}{]}\PY{p}{)}
            \PY{n}{H} \PY{o}{=} \PY{n}{Hmatrix}\PY{p}{(}\PY{n}{w}\PY{p}{[}\PY{o}{\PYZhy{}}\PY{l+m+mi}{1}\PY{p}{]}\PY{p}{)}
            \PY{n}{Hinv} \PY{o}{=} \PY{n}{np}\PY{o}{.}\PY{n}{linalg}\PY{o}{.}\PY{n}{inv}\PY{p}{(}\PY{n}{H}\PY{p}{)}
            \PY{n}{temp} \PY{o}{=} \PY{n}{w}\PY{p}{[}\PY{o}{\PYZhy{}}\PY{l+m+mi}{1}\PY{p}{]}\PY{o}{\PYZhy{}}\PY{n}{eta}\PY{o}{*}\PY{n}{Hinv}\PY{o}{.}\PY{n}{dot}\PY{p}{(}\PY{n}{g}\PY{p}{)}
            \PY{c+c1}{\PYZsh{} if counter k is larger than 5, the overall eta will be reduced by 1/2}
            \PY{n}{etanew} \PY{o}{=} \PY{n}{eta}
            \PY{n}{k} \PY{o}{=} \PY{l+m+mi}{0}
            \PY{c+c1}{\PYZsh{} if the temp is outside the D, will redo the while loop and add the counter}
            \PY{k}{while} \PY{p}{(}\PY{n+nb}{sum}\PY{p}{(}\PY{n}{temp}\PY{p}{)}\PY{o}{\PYZgt{}}\PY{o}{=}\PY{l+m+mi}{1} \PY{o+ow}{or} \PY{n}{temp}\PY{p}{[}\PY{l+m+mi}{0}\PY{p}{]}\PY{o}{\PYZlt{}}\PY{o}{=}\PY{l+m+mi}{0} \PY{o+ow}{or} \PY{n}{temp}\PY{p}{[}\PY{l+m+mi}{1}\PY{p}{]}\PY{o}{\PYZlt{}}\PY{o}{=}\PY{l+m+mi}{0}\PY{p}{)}\PY{p}{:}
                \PY{n}{etanew} \PY{o}{=} \PY{n}{etanew}\PY{o}{/}\PY{l+m+mi}{2}
                \PY{n}{temp} \PY{o}{=} \PY{n}{w}\PY{p}{[}\PY{o}{\PYZhy{}}\PY{l+m+mi}{1}\PY{p}{]}\PY{o}{\PYZhy{}}\PY{n}{etanew}\PY{o}{*}\PY{n}{Hinv}\PY{o}{.}\PY{n}{dot}\PY{p}{(}\PY{n}{g}\PY{p}{)}
                \PY{n}{k} \PY{o}{+}\PY{o}{=} \PY{l+m+mi}{1}
                \PY{n}{j} \PY{o}{+}\PY{o}{=} \PY{l+m+mi}{1}
            \PY{n}{w}\PY{o}{.}\PY{n}{append}\PY{p}{(}\PY{n}{temp}\PY{p}{)}
            \PY{n}{temploss} \PY{o}{=} \PY{n}{losfunction}\PY{p}{(}\PY{n}{temp}\PY{p}{)}
            \PY{n}{loss}\PY{o}{.}\PY{n}{append}\PY{p}{(}\PY{n}{temploss}\PY{p}{)}
            
            \PY{k}{if} \PY{n}{k}\PY{o}{\PYZgt{}}\PY{o}{=}\PY{l+m+mi}{5}\PY{p}{:}
                \PY{n}{eta} \PY{o}{=} \PY{n}{eta}\PY{o}{/}\PY{l+m+mi}{2}
        \PY{c+c1}{\PYZsh{} delete the dummy point}
        \PY{k}{del}\PY{p}{(}\PY{n}{w}\PY{p}{[}\PY{l+m+mi}{0}\PY{p}{]}\PY{p}{)}
        \PY{k}{del}\PY{p}{(}\PY{n}{loss}\PY{p}{[}\PY{l+m+mi}{0}\PY{p}{]}\PY{p}{)}
        \PY{c+c1}{\PYZsh{} convert w to ndarray for plot}
        \PY{n}{w} \PY{o}{=} \PY{n}{np}\PY{o}{.}\PY{n}{asarray}\PY{p}{(}\PY{n}{w}\PY{p}{)}
\end{Verbatim}

    \paragraph{\texorpdfstring{{Q. Report your initial point w0 and η of
your
choice.}}{Q. Report your initial point w0 and η of your choice.}}\label{q.-report-your-initial-point-w0-and-ux3b7-of-your-choice.}

\paragraph{\texorpdfstring{{The intial point is at {[} 0.2814473 ,
0.07728957{]}. The intial eta is 1, but if the update is out of
boundaries, eta will temporarily shrink by 1/2 until the update is
within boundaries. Also, if it takes for more than 5 temporaty attempts
to shrink the eta for one iteration, the overall eta will be shrinked by
1/2.However, unlike the gradient descendent, eta stays at 1.
}}{The intial point is at {[} 0.2814473 , 0.07728957{]}. The intial eta is 1, but if the update is out of boundaries, eta will temporarily shrink by 1/2 until the update is within boundaries. Also, if it takes for more than 5 temporaty attempts to shrink the eta for one iteration, the overall eta will be shrinked by 1/2.However, unlike the gradient descendent, eta stays at 1. }}\label{the-intial-point-is-at-0.2814473-0.07728957.-the-intial-eta-is-1-but-if-the-update-is-out-of-boundaries-eta-will-temporarily-shrink-by-12-until-the-update-is-within-boundaries.-also-if-it-takes-for-more-than-5-temporaty-attempts-to-shrink-the-eta-for-one-iteration-the-overall-eta-will-be-shrinked-by-12.however-unlike-the-gradient-descendent-eta-stays-at-1.}

    \paragraph{\texorpdfstring{{Q. Draw a graph that shows the trajectory
followed by the points at each iteration.
}}{Q. Draw a graph that shows the trajectory followed by the points at each iteration. }}\label{q.-draw-a-graph-that-shows-the-trajectory-followed-by-the-points-at-each-iteration.}

    \begin{Verbatim}[commandchars=\\\{\}]
{\color{incolor}In [{\color{incolor}9}]:} \PY{n}{fig}\PY{p}{,} \PY{n}{ax} \PY{o}{=} \PY{n}{plt}\PY{o}{.}\PY{n}{subplots}\PY{p}{(}\PY{p}{)}
        \PY{n}{A} \PY{o}{=} \PY{n}{w}\PY{p}{[}\PY{p}{:}\PY{p}{,}\PY{l+m+mi}{0}\PY{p}{]}
        \PY{n}{B} \PY{o}{=} \PY{n}{w}\PY{p}{[}\PY{p}{:}\PY{p}{,}\PY{l+m+mi}{1}\PY{p}{]}
        \PY{n}{n} \PY{o}{=} \PY{n+nb}{range}\PY{p}{(}\PY{n+nb}{len}\PY{p}{(}\PY{n}{w}\PY{p}{)}\PY{p}{)}
        \PY{n}{plt}\PY{o}{.}\PY{n}{plot}\PY{p}{(}\PY{n}{A}\PY{p}{,}\PY{n}{B}\PY{p}{,}\PY{l+s+s1}{\PYZsq{}}\PY{l+s+s1}{y\PYZhy{}\PYZgt{}}\PY{l+s+s1}{\PYZsq{}}\PY{p}{)}
        \PY{n}{C} \PY{o}{=} \PY{n+nb}{list}\PY{p}{(}\PY{n}{A}\PY{p}{)}
        \PY{k}{for} \PY{n}{i}\PY{p}{,} \PY{n}{txt} \PY{o+ow}{in} \PY{n+nb}{enumerate}\PY{p}{(}\PY{n}{C}\PY{p}{)}\PY{p}{:}
            \PY{n}{ax}\PY{o}{.}\PY{n}{annotate}\PY{p}{(}\PY{n}{n}\PY{p}{[}\PY{n}{i}\PY{p}{]}\PY{p}{,} \PY{p}{(}\PY{n}{A}\PY{p}{[}\PY{n}{i}\PY{p}{]}\PY{p}{,}\PY{n}{B}\PY{p}{[}\PY{n}{i}\PY{p}{]}\PY{p}{)}\PY{p}{)}
        \PY{n}{plt}\PY{o}{.}\PY{n}{plot}\PY{p}{(}\PY{p}{[}\PY{l+m+mi}{0}\PY{p}{,}\PY{l+m+mi}{1}\PY{p}{]}\PY{p}{,}\PY{p}{[}\PY{l+m+mi}{1}\PY{p}{,}\PY{l+m+mi}{0}\PY{p}{]}\PY{p}{,}\PY{l+s+s1}{\PYZsq{}}\PY{l+s+s1}{k\PYZhy{}}\PY{l+s+s1}{\PYZsq{}}\PY{p}{)}
        \PY{n}{plt}\PY{o}{.}\PY{n}{plot}\PY{p}{(}\PY{p}{[}\PY{l+m+mi}{0}\PY{p}{,}\PY{l+m+mi}{1}\PY{p}{]}\PY{p}{,}\PY{p}{[}\PY{l+m+mi}{0}\PY{p}{,}\PY{l+m+mi}{0}\PY{p}{]}\PY{p}{,}\PY{l+s+s1}{\PYZsq{}}\PY{l+s+s1}{k\PYZhy{}}\PY{l+s+s1}{\PYZsq{}}\PY{p}{)}
        \PY{n}{plt}\PY{o}{.}\PY{n}{plot}\PY{p}{(}\PY{p}{[}\PY{l+m+mi}{0}\PY{p}{,}\PY{l+m+mi}{0}\PY{p}{]}\PY{p}{,}\PY{p}{[}\PY{l+m+mi}{0}\PY{p}{,}\PY{l+m+mi}{1}\PY{p}{]}\PY{p}{,}\PY{l+s+s1}{\PYZsq{}}\PY{l+s+s1}{k\PYZhy{}}\PY{l+s+s1}{\PYZsq{}}\PY{p}{)}
        \PY{n}{plt}\PY{o}{.}\PY{n}{xlabel}\PY{p}{(}\PY{l+s+s1}{\PYZsq{}}\PY{l+s+s1}{x}\PY{l+s+s1}{\PYZsq{}}\PY{p}{)}
        \PY{n}{plt}\PY{o}{.}\PY{n}{title}\PY{p}{(}\PY{l+s+s1}{\PYZsq{}}\PY{l+s+s1}{Trajectory followed by the points at each iteration (c)}\PY{l+s+s1}{\PYZsq{}}\PY{p}{)}
        \PY{n}{plt}\PY{o}{.}\PY{n}{ylabel}\PY{p}{(}\PY{l+s+s1}{\PYZsq{}}\PY{l+s+s1}{y}\PY{l+s+s1}{\PYZsq{}}\PY{p}{)}
        \PY{n}{ax}\PY{o}{.}\PY{n}{text}\PY{p}{(}\PY{o}{.}\PY{l+m+mi}{6}\PY{p}{,} \PY{o}{.}\PY{l+m+mi}{8}\PY{p}{,} \PY{l+s+s1}{\PYZsq{}}\PY{l+s+s1}{\PYZdl{}}\PY{l+s+s1}{\PYZbs{}}\PY{l+s+s1}{epsilon =5e\PYZhy{}3\PYZdl{}}\PY{l+s+s1}{\PYZsq{}}\PY{p}{,} \PY{n}{style}\PY{o}{=}\PY{l+s+s1}{\PYZsq{}}\PY{l+s+s1}{italic}\PY{l+s+s1}{\PYZsq{}}\PY{p}{,}
                \PY{n}{bbox}\PY{o}{=}\PY{p}{\PYZob{}}\PY{l+s+s1}{\PYZsq{}}\PY{l+s+s1}{facecolor}\PY{l+s+s1}{\PYZsq{}}\PY{p}{:}\PY{l+s+s1}{\PYZsq{}}\PY{l+s+s1}{red}\PY{l+s+s1}{\PYZsq{}}\PY{p}{,} \PY{l+s+s1}{\PYZsq{}}\PY{l+s+s1}{alpha}\PY{l+s+s1}{\PYZsq{}}\PY{p}{:}\PY{l+m+mf}{0.3}\PY{p}{,} \PY{l+s+s1}{\PYZsq{}}\PY{l+s+s1}{pad}\PY{l+s+s1}{\PYZsq{}}\PY{p}{:}\PY{l+m+mi}{10}\PY{p}{\PYZcb{}}\PY{p}{,}\PY{n}{fontsize} \PY{o}{=} \PY{l+m+mi}{10}\PY{p}{)}
        \PY{n}{plt}\PY{o}{.}\PY{n}{show}\PY{p}{(}\PY{p}{)}
\end{Verbatim}

    \begin{center}
    \adjustimage{max size={0.9\linewidth}{0.9\paperheight}}{output_21_0.png}
    \end{center}
    { \hspace*{\fill} \\}
    
    \paragraph{\texorpdfstring{{Q. Plot the energies f(w0), f(w1), . . . ,
achieved by the points at each iteration.
}}{Q. Plot the energies f(w0), f(w1), . . . , achieved by the points at each iteration. }}\label{q.-plot-the-energies-fw0-fw1-.-.-.-achieved-by-the-points-at-each-iteration.}

    \begin{Verbatim}[commandchars=\\\{\}]
{\color{incolor}In [{\color{incolor}10}]:} \PY{n}{fig}\PY{p}{,} \PY{n}{ax} \PY{o}{=} \PY{n}{plt}\PY{o}{.}\PY{n}{subplots}\PY{p}{(}\PY{p}{)}
         \PY{n}{A} \PY{o}{=} \PY{n}{loss}
         \PY{n}{n} \PY{o}{=} \PY{n+nb}{range}\PY{p}{(}\PY{n+nb}{len}\PY{p}{(}\PY{n}{A}\PY{p}{)}\PY{p}{)}
         \PY{n}{plt}\PY{o}{.}\PY{n}{plot}\PY{p}{(}\PY{n}{n}\PY{p}{,}\PY{n}{A}\PY{p}{,}\PY{l+s+s1}{\PYZsq{}}\PY{l+s+s1}{r\PYZhy{}\PYZgt{}}\PY{l+s+s1}{\PYZsq{}}\PY{p}{)}
         \PY{k}{for} \PY{n}{i}\PY{p}{,} \PY{n}{txt} \PY{o+ow}{in} \PY{n+nb}{enumerate}\PY{p}{(}\PY{n}{A}\PY{p}{)}\PY{p}{:}
             \PY{n}{ax}\PY{o}{.}\PY{n}{annotate}\PY{p}{(}\PY{n+nb}{round}\PY{p}{(}\PY{n}{txt}\PY{p}{,}\PY{l+m+mi}{5}\PY{p}{)}\PY{p}{,} \PY{p}{(}\PY{n}{n}\PY{p}{[}\PY{n}{i}\PY{p}{]}\PY{p}{,}\PY{n}{A}\PY{p}{[}\PY{n}{i}\PY{p}{]}\PY{p}{)}\PY{p}{)}
         \PY{n}{plt}\PY{o}{.}\PY{n}{xlabel}\PY{p}{(}\PY{l+s+s1}{\PYZsq{}}\PY{l+s+s1}{Iteration}\PY{l+s+s1}{\PYZsq{}}\PY{p}{)}
         \PY{n}{plt}\PY{o}{.}\PY{n}{title}\PY{p}{(}\PY{l+s+s1}{\PYZsq{}}\PY{l+s+s1}{Cost at each iteration (c)}\PY{l+s+s1}{\PYZsq{}}\PY{p}{)}
         \PY{n}{plt}\PY{o}{.}\PY{n}{ylabel}\PY{p}{(}\PY{l+s+s1}{\PYZsq{}}\PY{l+s+s1}{Cost}\PY{l+s+s1}{\PYZsq{}}\PY{p}{)}
         \PY{n}{plt}\PY{o}{.}\PY{n}{show}\PY{p}{(}\PY{p}{)}
\end{Verbatim}

    \begin{center}
    \adjustimage{max size={0.9\linewidth}{0.9\paperheight}}{output_23_0.png}
    \end{center}
    { \hspace*{\fill} \\}
    
    \begin{Verbatim}[commandchars=\\\{\}]
{\color{incolor}In [{\color{incolor}11}]:} \PY{c+c1}{\PYZsh{} converging point}
         \PY{n}{w}\PY{p}{[}\PY{o}{\PYZhy{}}\PY{l+m+mi}{1}\PY{p}{]}
\end{Verbatim}

            \begin{Verbatim}[commandchars=\\\{\}]
{\color{outcolor}Out[{\color{outcolor}11}]:} array([ 0.33385717,  0.33241947])
\end{Verbatim}
        
    \begin{Verbatim}[commandchars=\\\{\}]
{\color{incolor}In [{\color{incolor}12}]:} \PY{c+c1}{\PYZsh{} converging cost}
         \PY{n}{loss}\PY{p}{[}\PY{o}{\PYZhy{}}\PY{l+m+mi}{1}\PY{p}{]}
\end{Verbatim}

            \begin{Verbatim}[commandchars=\\\{\}]
{\color{outcolor}Out[{\color{outcolor}12}]:} 3.2958425486216987
\end{Verbatim}
        
    \paragraph{(d) Compare the speed of convergence of gradient descent and
Newton's method, i.e. how fast does each method approach the estimated
global
minimum?}\label{d-compare-the-speed-of-convergence-of-gradient-descent-and-newtons-method-i.e.-how-fast-does-each-method-approach-the-estimated-global-minimum}

     With the same intialized starting point and eta, and the same
convergence condition, it took gradient descent \textbf{21} epochs while
\textbf{5} for the Newton's method. Meanwhile,this was accomplished with
the "smart" shrinkage programming on the eta on both methods. In the
gradient descent case, the gloabal eta shrunk down to 0.0625, while the
global eta of Newton's method stayed at 1 for the whole time. Finally,
the estimated global minimum is 3.303106 (gradient descent) vs. 3.295843
(Newton's method). In sum, with the appropriate choice of \(\eta\) if
the computation of the Hessian matrix and its inverse is not computation
heavy, Newton's method can achieve the covergence faster and probably
provide a more optimized cost function. It should be noted that a
oversmall or overbig \(\eta\) might make the Newton's method perform
worse than the gradient descent method.

    \subsubsection{Q.3}\label{q.3}

    \subparagraph{(a) Let xi = i, i = 1,...,50.}\label{a-let-xi-i-i-1...50.}

\subparagraph{(b) Let yi = i + ui, i = 1,...,50, where each ui should be
chosen to be an arbitrary real number between −1 and
1.}\label{b-let-yi-i-ui-i-1...50-where-each-ui-should-be-chosen-to-be-an-arbitrary-real-number-between-1-and-1.}

    \begin{Verbatim}[commandchars=\\\{\}]
{\color{incolor}In [{\color{incolor}13}]:} \PY{n}{np}\PY{o}{.}\PY{n}{random}\PY{o}{.}\PY{n}{seed}\PY{p}{(}\PY{l+m+mi}{45}\PY{p}{)}
         \PY{n}{x} \PY{o}{=} \PY{n+nb}{list}\PY{p}{(}\PY{n+nb}{range}\PY{p}{(}\PY{l+m+mi}{1}\PY{p}{,}\PY{l+m+mi}{51}\PY{p}{)}\PY{p}{)}
         \PY{n}{x} \PY{o}{=} \PY{n}{np}\PY{o}{.}\PY{n}{asarray}\PY{p}{(}\PY{n}{x}\PY{p}{)}
         \PY{n}{y} \PY{o}{=} \PY{n}{np}\PY{o}{.}\PY{n}{random}\PY{o}{.}\PY{n}{uniform}\PY{p}{(}\PY{o}{\PYZhy{}}\PY{l+m+mi}{1}\PY{p}{,}\PY{l+m+mi}{1}\PY{p}{,}\PY{l+m+mi}{50}\PY{p}{)} \PY{o}{+} \PY{n}{x}
\end{Verbatim}

    \paragraph{(c) Find the linear least squares fit to (xi , yi ), i = 1, .
. . , 50. Note that the linear least squares fit is the line y = w0 +
w1x, where w0 and w1 should be chosen to minimize 50 (yi − (w0 +
w1xi))2.}\label{c-find-the-linear-least-squares-fit-to-xi-yi-i-1-.-.-.-50.-note-that-the-linear-least-squares-fit-is-the-line-y-w0-w1x-where-w0-and-w1-should-be-chosen-to-minimize-50-yi-w0-w1xi2.}

{ Let
\(W = [w_0, w_1], X = \begin{bmatrix}1 & 1 & \cdots & 1 \\ x_1 & x_2 & \cdots & x_{50} \end{bmatrix}, Y = \begin{bmatrix} y_1 & y_2 & \cdots & y_{50} \end{bmatrix}\).\\
Thus, we are essentially finding \(\underset{W}{\arg\min}||Y-WX||^2\).
And we have already known that the solution to \(Y-WX=0\) is
\(W^*_{L.S.} = YX^+\), where \(X^+ = X^T(XX^T)^{-1}\) given X is full
row. Therefore, \(W^*_{L.S.} = YX^T(XX^T)^{-1}\) }

    \begin{Verbatim}[commandchars=\\\{\}]
{\color{incolor}In [{\color{incolor}14}]:} \PY{n}{Y} \PY{o}{=} \PY{n}{y}
         \PY{n}{X} \PY{o}{=} \PY{n}{np}\PY{o}{.}\PY{n}{vstack}\PY{p}{(}\PY{p}{(}\PY{n}{np}\PY{o}{.}\PY{n}{repeat}\PY{p}{(}\PY{l+m+mi}{1}\PY{p}{,}\PY{l+m+mi}{50}\PY{p}{)}\PY{p}{,}\PY{n}{x}\PY{p}{)}\PY{p}{)}
         \PY{n}{Wstar} \PY{o}{=} \PY{n}{Y}\PY{o}{.}\PY{n}{dot}\PY{p}{(}\PY{n}{np}\PY{o}{.}\PY{n}{transpose}\PY{p}{(}\PY{n}{X}\PY{p}{)}\PY{p}{)}\PY{o}{.}\PY{n}{dot}\PY{p}{(}\PY{n}{np}\PY{o}{.}\PY{n}{linalg}\PY{o}{.}\PY{n}{inv}\PY{p}{(}\PY{n}{X}\PY{o}{.}\PY{n}{dot}\PY{p}{(}\PY{n}{np}\PY{o}{.}\PY{n}{transpose}\PY{p}{(}\PY{n}{X}\PY{p}{)}\PY{p}{)}\PY{p}{)}\PY{p}{)}
         \PY{n}{Wstar}
\end{Verbatim}

            \begin{Verbatim}[commandchars=\\\{\}]
{\color{outcolor}Out[{\color{outcolor}14}]:} array([-0.049585  ,  0.99941613])
\end{Verbatim}
        
    { So w0 = -0.049585, w1 = 0.99941613 are to chosen to minimize E.}

    \paragraph{(d) Plot the points (xi; yi); i = 1; : : : ; 50 together with
their linear least squares
fit.}\label{d-plot-the-points-xi-yi-i-1-50-together-with-their-linear-least-squares-fit.}

    \begin{Verbatim}[commandchars=\\\{\}]
{\color{incolor}In [{\color{incolor}15}]:} \PY{n}{plt}\PY{o}{.}\PY{n}{scatter}\PY{p}{(}\PY{n}{x}\PY{p}{,}\PY{n}{y}\PY{p}{,}\PY{n}{color} \PY{o}{=} \PY{l+s+s1}{\PYZsq{}}\PY{l+s+s1}{orange}\PY{l+s+s1}{\PYZsq{}}\PY{p}{)}
         \PY{n}{yfit} \PY{o}{=} \PY{p}{[}\PY{n}{Wstar}\PY{p}{[}\PY{l+m+mi}{0}\PY{p}{]} \PY{o}{+} \PY{n}{Wstar}\PY{p}{[}\PY{l+m+mi}{1}\PY{p}{]} \PY{o}{*} \PY{n}{xi} \PY{k}{for} \PY{n}{xi} \PY{o+ow}{in} \PY{n}{x}\PY{p}{]}
         \PY{n}{plt}\PY{o}{.}\PY{n}{plot}\PY{p}{(}\PY{n}{x}\PY{p}{,} \PY{n}{yfit}\PY{p}{,}\PY{n}{color} \PY{o}{=} \PY{l+s+s1}{\PYZsq{}}\PY{l+s+s1}{black}\PY{l+s+s1}{\PYZsq{}}\PY{p}{)}
         \PY{n}{plt}\PY{o}{.}\PY{n}{xlabel}\PY{p}{(}\PY{l+s+s1}{\PYZsq{}}\PY{l+s+s1}{x}\PY{l+s+s1}{\PYZsq{}}\PY{p}{)}
         \PY{n}{plt}\PY{o}{.}\PY{n}{title}\PY{p}{(}\PY{l+s+s1}{\PYZsq{}}\PY{l+s+s1}{Plot for 3(d)}\PY{l+s+s1}{\PYZsq{}}\PY{p}{)}
         \PY{n}{plt}\PY{o}{.}\PY{n}{ylabel}\PY{p}{(}\PY{l+s+s1}{\PYZsq{}}\PY{l+s+s1}{y}\PY{l+s+s1}{\PYZsq{}}\PY{p}{)}
         \PY{n}{plt}\PY{o}{.}\PY{n}{show}\PY{p}{(}\PY{p}{)}
\end{Verbatim}

    \begin{center}
    \adjustimage{max size={0.9\linewidth}{0.9\paperheight}}{output_35_0.png}
    \end{center}
    { \hspace*{\fill} \\}
    
    \paragraph{(e) Find (on paper) the gradient of 50 (yi −(w0 +w1xi))2
(derivatives with respect to w0 and
w1).}\label{e-find-on-paper-the-gradient-of-50-yi-w0-w1xi2-derivatives-with-respect-to-w0-and-w1.}

{ Let \(E = \sum_{i=1}^{50}(y_i-(w_0+w_1x_i))^2\).\\
Then
\(\nabla E = (\frac{\partial E}{\partial w_0},\frac{\partial E}{\partial w_1}) = (-2\sum_{i=1}^{50}(y_i-(w_0+w_1x_i)),-2\sum_{i=1}^{50}((y_i-(w_0+w_1x_i))x_i) = 2(Y-WX)X^T\).\\
Thus, the update will be: \(w \leftarrow w - \eta \nabla E\) }

    \paragraph{(f) (Re)find the linear least squares fit using the gradient
descent algorithm. Compare with
(c).}\label{f-refind-the-linear-least-squares-fit-using-the-gradient-descent-algorithm.-compare-with-c.}

    \begin{Verbatim}[commandchars=\\\{\}]
{\color{incolor}In [{\color{incolor}16}]:} \PY{n}{np}\PY{o}{.}\PY{n}{random}\PY{o}{.}\PY{n}{seed}\PY{p}{(}\PY{l+m+mi}{45}\PY{p}{)}
         \PY{c+c1}{\PYZsh{} define a cost function}
         \PY{k}{def} \PY{n+nf}{cost\PYZus{}function} \PY{p}{(}\PY{n}{w}\PY{p}{)}\PY{p}{:}
             \PY{n}{cost} \PY{o}{=} \PY{n}{np}\PY{o}{.}\PY{n}{sum}\PY{p}{(}\PY{p}{(}\PY{n}{Y}\PY{o}{\PYZhy{}}\PY{n}{w}\PY{o}{.}\PY{n}{dot}\PY{p}{(}\PY{n}{X}\PY{p}{)}\PY{p}{)}\PY{o}{*}\PY{o}{*}\PY{l+m+mi}{2}\PY{p}{)}
             \PY{k}{return} \PY{n}{cost}
         
         \PY{c+c1}{\PYZsh{}  setup}
         \PY{n}{w} \PY{o}{=} \PY{n+nb}{list}\PY{p}{(}\PY{p}{)}
         \PY{n}{cost} \PY{o}{=} \PY{n+nb}{list}\PY{p}{(}\PY{p}{)}
         \PY{n}{epsilon} \PY{o}{=} \PY{l+m+mf}{1e\PYZhy{}6}
         \PY{n}{eta} \PY{o}{=} \PY{l+m+mf}{3e\PYZhy{}4}
         \PY{n}{j} \PY{o}{=} \PY{l+m+mi}{0}
         
         \PY{c+c1}{\PYZsh{} give a dummy start point which will be deleted later so that index is working}
         \PY{n}{temp} \PY{o}{=} \PY{n}{np}\PY{o}{.}\PY{n}{array}\PY{p}{(}\PY{p}{[}\PY{l+m+mf}{0.1}\PY{p}{,}\PY{l+m+mf}{0.8}\PY{p}{]}\PY{p}{)}
         \PY{n}{temploss} \PY{o}{=} \PY{n}{cost\PYZus{}function}\PY{p}{(}\PY{n}{temp}\PY{p}{)}
         \PY{n}{w}\PY{o}{.}\PY{n}{append}\PY{p}{(}\PY{n}{temp}\PY{p}{)}
         \PY{n}{cost}\PY{o}{.}\PY{n}{append}\PY{p}{(}\PY{n}{temploss}\PY{p}{)}
         
         \PY{c+c1}{\PYZsh{} initialize the starting point w\PYZus{}0 in D}
         \PY{n}{temp} \PY{o}{=} \PY{n}{np}\PY{o}{.}\PY{n}{random}\PY{o}{.}\PY{n}{uniform}\PY{p}{(}\PY{l+m+mi}{0}\PY{p}{,}\PY{l+m+mi}{1}\PY{p}{,}\PY{l+m+mi}{2}\PY{p}{)}
         \PY{c+c1}{\PYZsh{} append the first point in w}
         \PY{n}{w}\PY{o}{.}\PY{n}{append}\PY{p}{(}\PY{n}{temp}\PY{p}{)}
         \PY{n}{temploss} \PY{o}{=} \PY{n}{cost\PYZus{}function}\PY{p}{(}\PY{n}{temp}\PY{p}{)}
         \PY{n}{cost}\PY{o}{.}\PY{n}{append}\PY{p}{(}\PY{n}{temploss}\PY{p}{)}
         
         \PY{c+c1}{\PYZsh{} the while loops ends when converge}
         \PY{k}{while} \PY{p}{(}\PY{n}{np}\PY{o}{.}\PY{n}{abs}\PY{p}{(}\PY{n}{cost}\PY{p}{[}\PY{o}{\PYZhy{}}\PY{l+m+mi}{1}\PY{p}{]}\PY{o}{\PYZhy{}}\PY{n}{cost}\PY{p}{[}\PY{o}{\PYZhy{}}\PY{l+m+mi}{2}\PY{p}{]}\PY{p}{)}\PY{o}{\PYZgt{}}\PY{o}{=}\PY{n}{epsilon}\PY{p}{)}\PY{p}{:}
             \PY{c+c1}{\PYZsh{} find the gradient}
             \PY{n}{g} \PY{o}{=} \PY{n}{np}\PY{o}{.}\PY{n}{array}\PY{p}{(}\PY{p}{[}\PY{o}{\PYZhy{}}\PY{l+m+mi}{2}\PY{o}{*}\PY{n}{np}\PY{o}{.}\PY{n}{sum}\PY{p}{(}\PY{n}{Y}\PY{o}{\PYZhy{}}\PY{n}{w}\PY{p}{[}\PY{o}{\PYZhy{}}\PY{l+m+mi}{1}\PY{p}{]}\PY{o}{.}\PY{n}{dot}\PY{p}{(}\PY{n}{X}\PY{p}{)}\PY{p}{)}\PY{p}{,}\PY{o}{\PYZhy{}}\PY{l+m+mi}{2}\PY{o}{*}\PY{n}{np}\PY{o}{.}\PY{n}{sum}\PY{p}{(}\PY{p}{(}\PY{n}{Y}\PY{o}{\PYZhy{}}\PY{n}{w}\PY{p}{[}\PY{o}{\PYZhy{}}\PY{l+m+mi}{1}\PY{p}{]}\PY{o}{.}\PY{n}{dot}\PY{p}{(}\PY{n}{X}\PY{p}{)}\PY{p}{)} \PY{o}{*} \PY{n}{x}\PY{p}{)}\PY{p}{]}\PY{p}{)}\PY{o}{/}\PY{l+m+mi}{50}
             \PY{n}{temp} \PY{o}{=} \PY{n}{w}\PY{p}{[}\PY{o}{\PYZhy{}}\PY{l+m+mi}{1}\PY{p}{]}\PY{o}{\PYZhy{}}\PY{n}{eta}\PY{o}{*}\PY{n}{g}
             \PY{n}{w}\PY{o}{.}\PY{n}{append}\PY{p}{(}\PY{n}{temp}\PY{p}{)}
             \PY{n}{temploss} \PY{o}{=} \PY{n}{cost\PYZus{}function}\PY{p}{(}\PY{n}{temp}\PY{p}{)}
             \PY{n}{cost}\PY{o}{.}\PY{n}{append}\PY{p}{(}\PY{n}{temploss}\PY{p}{)}
             \PY{n}{j} \PY{o}{+}\PY{o}{=} \PY{l+m+mi}{1}
         
         \PY{c+c1}{\PYZsh{} delete the dummy point}
         \PY{k}{del}\PY{p}{(}\PY{n}{w}\PY{p}{[}\PY{l+m+mi}{0}\PY{p}{]}\PY{p}{)}
         \PY{k}{del}\PY{p}{(}\PY{n}{cost}\PY{p}{[}\PY{l+m+mi}{0}\PY{p}{]}\PY{p}{)}
         \PY{c+c1}{\PYZsh{} convert w to ndarray for plot}
         \PY{n}{w} \PY{o}{=} \PY{n}{np}\PY{o}{.}\PY{n}{asarray}\PY{p}{(}\PY{n}{w}\PY{p}{)}
         \PY{n}{j}
\end{Verbatim}

            \begin{Verbatim}[commandchars=\\\{\}]
{\color{outcolor}Out[{\color{outcolor}16}]:} 28427
\end{Verbatim}
        
    \begin{Verbatim}[commandchars=\\\{\}]
{\color{incolor}In [{\color{incolor}17}]:} \PY{n}{w}\PY{p}{[}\PY{o}{\PYZhy{}}\PY{l+m+mi}{1}\PY{p}{]}
\end{Verbatim}

            \begin{Verbatim}[commandchars=\\\{\}]
{\color{outcolor}Out[{\color{outcolor}17}]:} array([-0.03274961,  0.99891593])
\end{Verbatim}
        
    \begin{Verbatim}[commandchars=\\\{\}]
{\color{incolor}In [{\color{incolor}18}]:} \PY{n}{cost\PYZus{}function}\PY{p}{(}\PY{n}{Wstar}\PY{p}{)}
\end{Verbatim}

            \begin{Verbatim}[commandchars=\\\{\}]
{\color{outcolor}Out[{\color{outcolor}18}]:} 16.526723550620243
\end{Verbatim}
        
    \begin{Verbatim}[commandchars=\\\{\}]
{\color{incolor}In [{\color{incolor}19}]:} \PY{n}{cost}\PY{p}{[}\PY{o}{\PYZhy{}}\PY{l+m+mi}{1}\PY{p}{]}
\end{Verbatim}

            \begin{Verbatim}[commandchars=\\\{\}]
{\color{outcolor}Out[{\color{outcolor}19}]:} 16.530161196220153
\end{Verbatim}
        
    \paragraph{\texorpdfstring{{ Compare with
(c).}}{ Compare with (c).}}\label{compare-with-c.}

{So w0 = -0.03275, w1 = 0.99892 are to chosen to by the gradient descent
method after 28427 epoch (batch version), compared with the w0 =
-0.049585, w1 = 0.99941613 chosen by the least square mothod in one
iteration. The cost by gradient descent is 16.53016, while the least
square's cost is 16.52672. And an appropriate choice of \(\eta\) for
gradient descent is tricky.}

    \textbf{(g) Show (on paper) that a single iteration of Newton's
method with η = 1 provides the globally optimal solution (the solution
in (c)) regardless of the initial
point.}\\

{ From (c), we have already known that the least square
\(W^*_{L.S.} = YX^T(XX^T)^{-1}\).\\
By gradient descent method, the gradient is:
\(\nabla E = \frac{\partial (Y-WX)(Y-WX)^T}{\partial W} = 2(Y-WX)X^T\).\\
And the Hessian matrix is:
\(H = \frac{\partial \nabla E}{\partial W} = \frac{\partial 2(Y-WX)X^T}{\partial W} =-2XX^T\),
and \(H^{-1} = -\frac{1}{2}(XX^T)^{-1}\).\\
Thus,
\(W^*_{G.D.} = W - \eta \nabla E H^{-1} = W + (Y-WX)X^T(XX^T)^{-1} = W + YX^T(XX^T)^{-1} - W = YX^T(XX^T)^{-1} = W^*_{L.S.}\)
}


    % Add a bibliography block to the postdoc
    
    
    
    \end{document}
